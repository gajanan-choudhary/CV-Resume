%--------------------------------%
% Author     : Gajanan Choudhary %
% License    : MIT               %
% About file : LaTeX Resume      %
%--------------------------------%

\documentclass[letterpaper,10pt]{article}

%--------------------------------%
% Author     : Gajanan Choudhary %
% License    : MIT               %
% About file : LaTeX Resume      %
%--------------------------------%

\RequirePackage{ifpdf}[2.3]
\ifpdf
  \pdfminorversion=7
\fi

\usepackage{cmap}  % For fixing copy pasting ligatures - ff, fi, fl, ffl, etc.
%Partially deal with overfull hbox / vbox
\usepackage[T1]{fontenc}
\usepackage[final]{microtype}
\usepackage{lmodern}
%\usepackage{ebgaramond}

% Some packages to write mathematics.
%\usepackage{amsmath,amsthm,amsfonts,amscd} 
%\usepackage{eucal} 	 	% Euler fonts
%\usepackage{epsfig}    % Allows inclusion of eps files.
%\usepackage{citesort}   % 

\usepackage{url} 		% Allows good typesetting of web URLs.

\usepackage{latexsym}
%\usepackage{graphics}
%\usepackage{epstopdf}
%\usepackage{algorithmic}
\usepackage{amssymb}
\usepackage{booktabs, makecell}
%\usepackage{bm}
%\usepackage[section]{placeins}
%\usepackage{xfp}
\usepackage{enumitem}  % For indentation of bullets

%%%%%%%%%%%%%%%%%%%%%%%%%%%%%%%%%%%%%%%%%%%%%%%%%%%%%%%%%%%%%%%%%%%%%%
\usepackage[referable]{threeparttablex}
\usepackage{footnote}
%\usepackage{cite} %% Had to comment out to prevent error
%\usepackage{siamcleveref}       %

%----------------------------------%
% Need the following for Wordcloud %
%----------------------------------%
\usepackage{graphicx}         	% Allows inclusion of eps files.
\usepackage{tikz}
\usepackage{tikzscale}
%\usepackage[percent]{overpic}
%\usetikzlibrary{calc}
%\usetikzlibrary{3d}
\usepackage{wrapfig}
\usepackage[utf8]{inputenc}
\usepackage{anyfontsize}
%\usepackage[export]{adjustbox} % Bounding box on figure
\DeclareGraphicsExtensions{.tikz,.pdf} %.eps,.png,.jpg,

\usepackage{contour}
\usepackage{ulem}

%\usepackage{csquotes}             % idk, babel/biber like it
%\usepackage[style=apa,
%backend=biber,
%sortcites=true,
%sorting=ydmdnt,
%language=american]{biblatex}      % for reference sections
%

%----------------------------------------------------------------%
% Packages borrowed from (MIT licensed) github.com/sb2nov/resume %
%----------------------------------------------------------------%
\usepackage[empty]{fullpage}
\usepackage{titlesec}
\usepackage{marvosym}
%\usepackage[usenames,dvipsnames]{color}
\usepackage{verbatim}
\usepackage{enumitem}
\usepackage[hidelinks]{hyperref}
\usepackage{fancyhdr}
\usepackage[english]{babel}
\usepackage{tabularx}
%-----------------------------------------------------------------%


%--------------------------------%
% Author     : Gajanan Choudhary %
% License    : MIT               %
% About file : LaTeX Resume      %
%--------------------------------%

%----------------------------------------------%
% Edit the following -- Put your details below %
%----------------------------------------------%

\def\Author{Gajanan Choudhary}
\def\AuthorEmail{gajanan@utexas.edu}
\def\AuthorAddress{Lake Austin Blvd, Austin, TX 78703, USA}
\def\AuthorPhoneText{\smallplus{}1 (512) 657-3030}
\def\AuthorPhoneLink{tel:15126573030}

% You need to escape special characters in website link.
% See: https://stackoverflow.com/questions/2894710/how-to-write-urls-in-latex
\def\AuthorWebsiteText{https://users.oden.utexas.edu/\textasciitilde{}gajanan/}
\def\AuthorWebsiteLink{https://users.oden.utexas.edu/~gajanan/}

%%%----------------------------------------------------------%%%
%%%----------------------------------------------------------%%%
%%% Most likely, you will not need to edit things below this %%%
%%%----------------------------------------------------------%%%
%%%----------------------------------------------------------%%%

%------------------%
% Utility commands %
%------------------%

%---------------------%
% Use sans-serif font %
%---------------------%
%\renewcommand{\familydefault}{\sfdefault}

%--------------%
% Line spacing %
%--------------%
\linespread{1.2}

%-------%
% Dates %
%-------%
\newcommand{\dates}[2]{#1 -- #2}

%---------------------------%
% For underlining hyperlink %
%---------------------------%
\renewcommand{\ULdepth}{1.8pt}
\contourlength{0.8pt}
\newcommand{\myuline}[1]{%
  \uline{\phantom{#1}}%
  \llap{\contour{white}{#1}}%
}

%-----------%
% Wordcloud %
%-----------%
\definecolor{uto}{RGB}{191, 87, 0}
\definecolor{utg}{RGB}{51, 63, 72}

% The following is needed for properly scaling text in the wordcloud. See:
% https://tex.stackexchange.com/questions/328228/appearance-of-tiny-or-scriptsize-fontsize-in-latex-horizontal-stretch/328237
%%% TEXT FONT FIX
\rmfamily
\DeclareFontShape{T1}{lmr}{bx}{sc}{<-> cmr10}{}% USE BOLD SCSHAPE NOT OTHERWISE DEFINED
%%% MATH FONT FIX
\DeclareFontFamily{OML}{zlmm}{}
\DeclareFontShape{OML}{zlmm}{m}{it}{<-> lmmi10}{}
\DeclareFontShape{OML}{zlmm}{b}{it}{<->ssub * zlmm/m/it}{}
\DeclareFontShape{OML}{zlmm}{bx}{it}{<->ssub * zlmm/m/it}{}

\DeclareMathVersion{Tinyb}
\SetSymbolFont{operators}{Tinyb}{T1}{lmr}{bx}{sc}
\SetSymbolFont{letters}{Tinyb}{OML}{zlmm}{m}{it}
%%%
\newenvironment{tinyb}{\bgroup\tiny\bfseries\scshape\mathversion{Tinyb}}{\egroup} % edit: now with \tiny included


%---------------%
% Writing "C++" %
%---------------%
% See: https://tex.stackexchange.com/questions/4302/prettiest-way-to-typeset-c-cplusplus
\def\CC{{C\nolinebreak[4]\hspace{-.05em}\raisebox{.4ex}{\tiny\bf ++}}}
% Needed for wordcloud:
\def\bigCC{{C\nolinebreak[4]\hspace{-.05em}\raisebox{.4ex}{\normalsize\bf ++}}}

%---------------------%
% Writing smaller "+" %
%---------------------%
\def\smallplus{\nolinebreak[2]\raisebox{.4ex}{\tiny\bf +}}

%---------------------------------------------------------------------------%
% Utility commands -- Borrowed from (MIT licensed) github.com/sb2nov/resume %
% Note: I have modified a few things after borrowing.                       %
%---------------------------------------------------------------------------%

\pagestyle{fancy}
\fancyhf{} % clear all header and footer fields
\fancyfoot{}
\renewcommand{\headrulewidth}{0pt}
\renewcommand{\footrulewidth}{0pt}

% Adjust margins
\addtolength{\oddsidemargin}{-0.25in}
\addtolength{\evensidemargin}{-0.25in}
\addtolength{\textwidth}{0.5in}
\addtolength{\topmargin}{-0.15in}
\addtolength{\textheight}{0.5in}

\urlstyle{same}

\raggedbottom
\raggedright
\setlength{\tabcolsep}{0in}

%---------------------------------------------------------------------------%
% Sections formatting
\titleformat{\section}{
  \vspace{-19pt}\scshape\raggedright\large
}{}{0em}{\vspace{-4pt}\color{gray}\titlerule[0.5pt]\color{black}\\}[\vspace{-6pt}]
%}{}{0em}{\color{gray}\rule{\linewidth}{0.5pt}\color{black}\\}[\vspace{-6pt}]


%---------------------------------------------------------------------------%
%---- gkc Modifying ----%
%\newcommand{\resumeSubheading}[4]{  % Original
\newcommand{\resumeSubheading}[2]{
  \vspace{-4pt}\item
    \begin{tabularx}{\linewidth}[t]{Xr}
      \textbf{#1} & #2 \\
    \end{tabularx}\vspace{-13pt}
}

\newcommand{\resumeSubSubheading}[2]{
    \vspace{-3pt}
    \begin{tabularx}{\linewidth}{Xr}
      #1 & #2 \\
    \end{tabularx}\vspace{-13pt}
}

%---------------------------------------------------------------------------%
\newcommand{\resumeItem}[2]{
  \item{
    {\textit{#1}}{: #2 \vspace{-2.5pt}}
  }
}

\newcommand{\resumeSubItem}[2]{
  \item{

    \textbf{#1}{: #2 \vspace{-7pt}} 
%    \textbf{#1}{: #2 \vspace{-9pt}}  % Use this if 11pt article
  }
}

\newcommand{\resumeUnlabeledItem}[1]{\item{{#1 \vspace{-2.5pt}}}}

%\newcommand{\resumeSubItem}[2]{\resumeItem{#1}{#2}\vspace{-4pt}}
%---------------------------------------------------------------------------%
% Bullet size and type
\renewcommand{\labelitemi}{\raisebox{1pt}{\footnotesize{$\bullet$}}}
\renewcommand{\labelitemii}{\raisebox{1pt}{\footnotesize{$\circ$}}}

%---------------------------------------------------------------------------%
\newcommand{\resumeSubHeadingListStart}{\begin{itemize}[leftmargin=0.225in]}
\newcommand{\resumeSubHeadingListEnd}{\end{itemize}\vspace{-3pt}}

%---------------------------------------------------------------------------%
\newcommand{\resumeSubItemListStart}{\vspace{-2pt}\begin{itemize}[leftmargin=0.225in]}
\newcommand{\resumeSubItemListEnd}{\end{itemize}\vspace{2pt}}

%---------------------------------------------------------------------------%
\newcommand{\resumeItemListStart}{\vspace{-7pt}\begin{itemize}[leftmargin=0.225in]}
\newcommand{\resumeItemListEnd}{\end{itemize}}

%---------------------------------------------------------------------------%
% Borrowed from:
%     https://github.com/btskinner/tex_cv
% If you want to go the partially manual route, look at another repo:
%     https://github.com/rgeirhos/academic-cv-publications

% ------------------------%
% BibLaTeX sorting scheme %
% ------------------------%

% Create a new sorting scheme that will sort in reverse chronological
% order, taking months into account correctly (i.e., in chronological
% order and not alphabetical order)

\DeclareSortingScheme{ydmdnt}{
  \sort{
    \field{presort}
  }
  \sort[final]{
    \field{sortkey}
  }
  \sort[direction=descending]{
    \field{sortyear}
    \field{year}
    \literal{9999}
  }
  \sort[direction=descending]{
    \field[padside=left,padwidth=2,padchar=0]{month}
    \literal{99}
  }
  \sort[direction=descending]{
    \field[padside=left,padwidth=2,padchar=0]{day}
    \literal{99}
  }
  \sort{
    \field{sortname}
    \field{author}
    \field{editor}
    \field{translator}
    \field{sorttitle}
    \field{title}
  }
  \sort{
    \field{sorttitle}
  }
  \sort[direction=descending]{
    \field[padside=left,padwidth=4,padchar=0]{volume}
    \literal{9999}
  }
}

%------------------%
% BibLaTeX options %
%------------------%

% language mapping
\DeclareLanguageMapping{american}{american-apa}

% ignore addendum and note fields that are sometimes useful, but not here
\DeclareSourcemap{
  \maps[datatype=bibtex]{
    \map{
      \step[fieldset=addendum, null]
      \step[fieldset=note, null]
    } 
  }
}

% macro so titles are converted to doi, url, or isbn link (if available)
% h\t https://tex.stackexchange.com/a/48409
\ExecuteBibliographyOptions{doi=false,url=false,isbn=false}
\newbibmacro{string+doiurlisbn}[1]{%
  \iffieldundef{doi}{%
    \iffieldundef{url}{%
      \iffieldundef{isbn}{%
        \iffieldundef{issn}{%
          #1%
        }{%
          \href{http://books.google.com/books?vid=ISSN\thefield{issn}}{#1}%
        }%
      }{%
        \href{http://books.google.com/books?vid=ISBN\thefield{isbn}}{#1}%
      }%
    }{%
      \href{\thefield{url}}{#1}%
    }%
  }{%
    \href{http://dx.doi.org/\thefield{doi}}{#1}%
  }%
}

\DeclareFieldFormat{title}{\usebibmacro{string+doiurlisbn}{\mkbibemph{#1}}}
\DeclareFieldFormat[article,incollection]{title}%
{\usebibmacro{string+doiurlisbn}{\mkbibquote{#1}}}

% no letter for same year entries
\defbibenvironment{mybib}
{\list
  {}
  {\setlength{\leftmargin}{\bibhang}%
    \setlength{\itemindent}{-\leftmargin}%
    \setlength{\itemsep}{\bibitemsep}%
    \setlength{\parsep}{\bibparsep}}}
{\endlist}
{\clearfield{extradate}\item}

% filter by year (only print those after year value in \recentyear
% macro above)
\defbibcheck{recent}{
  \iffieldint{year}
  {\ifnumless{\thefield{year}}{\recentyear}
    {\skipentry}
    {}}
  {\skipentry}
}



\begin{document}

%-----------------------------%
% Customize your resume below %
%-----------------------------%
\href{\AuthorWebsiteLink}{\scshape\huge\Author}\vspace{2pt}

\begin{small}
  %Website: \href{\AuthorWebsiteLink}{\color[HTML]{0f4cb4}\myuline{\smash{\AuthorWebsiteText}}}
  Website: \href{\AuthorWebsiteLink}{\AuthorWebsiteText}

  \vspace{2pt}\AuthorAddress

  \vspace{2pt}\href{mailto:\AuthorEmail}{\AuthorEmail} |
  \href{\AuthorPhoneLink}{\AuthorPhoneText}
\end{small}


%------------------------------------HEADING-----------------------------------%
%\setlength{\columnsep}{0pt}%
\vspace{-82pt}
\begin{wrapfigure}[2]{R}{0pt} %{-3cm}
  \centering
  %\includegraphics[width=0.33\linewidth]{wordcloud}
  %\begin{tikzpicture}[xscale=1, yscale=2, y=-1cm]
  %\begin{tikzpicture}[y=-1cm, every node/.style={right}]
  \begin{tikzpicture}[x=1in, y=-1in, every node/.style={right}] 
    %\node[color=uto]at (0.05,0.74){\scshape\huge Computational Scientist};
    \node[color=uto]at (0.10,0.74){\huge Computational Scientist\space };
    \begin{tinyb}
      %\node[color=uto]at (0.20,0.74){\huge Computational Scientist\space };
      \node[color=utg]at (1.36,0.61){\small Fortran\space };
      \node[color=utg]at (2.26,0.61){\scriptsize ERDC\space };
      \node[color=utg]at (2.20,0.12){\footnotesize OpenMP\space };
      \node[color=uto]at (1.24,0.14){\footnotesize Image Processing\space };
      \node[color=uto]at (0.75,0.19){\normalsize Docker\space };
      \node[color=utg]at (2.16,0.32){\small Travis CI\space };
      \node[color=utg]at (1.66,0.32){\footnotesize Coveralls\space };
      \node[color=uto]at (1.99,0.23){\scriptsize Make\space };
      \node[color=uto]at (2.39,0.21){\scriptsize UT Austin\space };
      \node[color=uto]at (1.07,0.35){\large GitHub\space };
      \node[color=utg]at (1.51,0.22){\tiny STAAD Pro\space };
      \node[color=utg]at (0.17,0.34){\Large Python\space };
      \node[color=uto]at (0.57,0.25){\tiny CSS\space };
      \node[color=uto]at (0.38,0.45){\scriptsize Matplotlib\space };
      \node[color=utg]at (0.85,0.43){\footnotesize Bash\space };
      \node[color=uto]at (0.26,0.13){\footnotesize ANSYS\space };
      
      \node[color=utg]at (2.49,0.42){\tiny AutoCAD\space };
      \node[color=uto]at (1.93,0.48){\normalsize Finite Elements\space };
      
      \node[color=utg]at (1.15,0.50){\scriptsize SVN\space };
      \node[color=uto]at (0.79,0.57){\footnotesize ctypes\space };
      \node[color=utg]at (0.42,0.58){\normalsize MPI\space };
      \node[color=utg]at (0.77,0.89){\footnotesize IIT Kharagpur\space };
      \node[color=uto]at (1.43,0.48){\footnotesize ADCIRC\space };
      \node[color=uto]at (1.70,0.92){\normalsize AdH\space };
      
      \node[color=uto]at (0.31,0.88){\footnotesize Abaqus\space };
      \node[color=utg]at (0.12,1.03){\footnotesize SWIG\space };
      
      \node[color=uto]at (0.50,1.03){\large CircleCI\space };
      \node[color=utg]at (0.65,1.17){\normalsize Neural Networks\space };
      \node[color=uto]at (1.31,1.11){\tiny HTML\space };
      \node[color=utg]at (2.05,1.03){\huge C/\bigCC{}\space };
      \node[color=uto]at (2.65,0.90){\scriptsize Git\space };
      \node[color=uto]at (0.16,1.18){\small CMake\space };
      \node[color=uto]at (2.40,1.19){\scriptsize SMS/WMS\space };
      \node[color=uto]at (1.76,1.18){\scriptsize ParaView\space };
      \node[color=utg]at (1.73,1.06){\small f2py\space };
      \node[color=uto]at (1.18,1.01){\footnotesize MATLAB\space };
      \draw[color=gray] (current bounding box.north east) -- (current bounding box.north west) -- (current bounding box.south west) -- (current bounding box.south east) -- cycle;
    \end{tinyb}
  \end{tikzpicture}
\end{wrapfigure}
\strut{} %% For ending wrapfig wrapping properly
\vspace{68pt}

% Following needed since wordcloud messes up wrapping the \hrule lines

%%\parbox{0.99\linewidth}{
%  %\begin{center}
%      \href{\AuthorWebsiteLink}{\scshape\huge\Author}\vspace{2pt}
%  %\end{center}
%%}% End of: \parbox{0.546\linewidth}{
%
%  %\begin{center}
%    \begin{small}
%      %Website: \href{\AuthorWebsiteLink}{\color[HTML]{0f4cb4}\myuline{\smash{\AuthorWebsiteText}}}
%      Website: \href{\AuthorWebsiteLink}{\AuthorWebsiteText}
%
%      \vspace{2pt}\AuthorAddress
%
%      \vspace{2pt}\href{mailto:\AuthorEmail}{\AuthorEmail} |
%      \href{\AuthorPhoneLink}{\AuthorPhoneText}
%
%    \end{small}
%  %\end{center}

%------------------------------------SUMMARY-----------------------------------%
\section{Summary}
\vspace{-2pt}
Computational scientist with 8 years of interdisciplinary research and
test-driven scientific software development experience through 15\smallplus{}
projects spanning high-performance computing (HPC), object-oriented programming,
applied mathematics, numerical methods, computational mechanics, optimization,
and machine learning. Expert in scientific software coupling and mixed-language
programming with C/\CC{}, Fortran, and Python.
% \\[1\baselineskip] added above since \parbox{0.546\linewidth}{\section{Summary}}
% messes up line spacing in the new section causing overlapping lines.

%------------------------------PROGRAMMING SKILLS------------------------------%
\section{Skills}
  \resumeSubItemListStart
    \resumeSubItem{Programming}{C/\CC{}, Fortran, Python, MATLAB, Bash, MPI,
                               OpenMP, f2py, SWIG, and Python/C API}
    \resumeSubItem{Tools}{Git, GitHub, Travis CI, CircleCI, Docker, Coveralls,
                         CMake, Make, Gcov, LCOV, Doxygen, and \LaTeX{}}
    \resumeSubItem{Engineering}{AdH, ADCIRC, GSSHA, Aquaveo SMS/WMS,
                                ANSYS, Abaqus, ParaView, and AutoCAD}
  \resumeSubItemListEnd

%----------------------------------EXPERIENCE----------------------------------%
\section{Experience}
  \resumeSubHeadingListStart

    %---------------------------------Job 1------------------------------------%
    \resumeSubheading
      {The University of Texas at Austin}{Austin, TX}

      %------------------------------Title 1-----------------------------------%
      \resumeSubSubheading
        {\textbf{Postdoctoral Fellow}}{\dates{October 2019}{Present}}
        \resumeItemListStart
          \resumeItem{Responsibilities}
            {Open-source parallel software development for research on coupled
            fluid dynamics models}
          \resumeItem{Key contribution}
            {Coupled diffusive wave and groundwater models in the \textbf{C}
            software, Adaptive Hydraulics (AdH), and created a \textbf{Python}
            HPC interface for the \textbf{Fortran} software, ADvanced
            CIRCulation (ADCIRC)}
            %{Developed a \textbf{Python} HPC interface for the \textbf{Fortran}
            %software, ADvanced CIRCulation (ADCIRC), and coupled groundwater and
            %diffusive wave models in the \textbf{C} software, Adaptive
            %Hydraulics (AdH)}
        \resumeItemListEnd

      %------------------------------Title 2-----------------------------------%
      \resumeSubSubheading
       {\textbf{Graduate Research Assistant}}{\dates{September 2014}{October 2019}}
       \resumeItemListStart
          \resumeItem{Responsibilities}
            {Software and library development for research on coupled
            fluid dynamics models}
          \resumeItem{Key contribution}
            {Coupled the \textbf{\CC{}} software, Gridded Surface Subsurface
            Hydrologic Analysis (GSSHA), with \textbf{C}-based AdH by
            designing, implementing, testing, validating, and coupling their
            \textbf{Python} interfaces}
            %{Designed, implemented, tested, and validated \textbf{Python} HPC
            %interfaces of \textbf{\CC{}} software, Gridded Surface Subsurface
            %Hydrologic Analysis (GSSHA), and the \textbf{C} software, AdH, and
            %coupled them}
       \resumeItemListEnd

    %---------------------------------Job 2------------------------------------%
    \resumeSubheading
      {Indian Register of Shipping}{Mumbai, India}

      %------------------------------Title 1-----------------------------------%
      \resumeSubSubheading
        {\textbf{Assistant Surveyor}}{\dates{July 2013}{July 2014}}
        \resumeItemListStart
          \resumeItem{Responsibilities}
            {Research on stress response of ship hulls to bending, shear,
            torsion, and warping loads}
          \resumeItem{Key contribution}
            {Developed a \textbf{MATLAB} software with a \textbf{GUI} for 2D
            modeling and stress analysis of ships}
        \resumeItemListEnd

    %%---------------------------------Job 3------------------------------------%
    %\resumeSubheading
    %  {Himanshu Tulpule and Associates}{Pune, India}

    %  %------------------------------Title 1-----------------------------------%
    %  \resumeSubSubheading
    %    {\textbf{Intern, Structural Design}}{\dates{May 2012}{July 2012}}

    %  %------------------------------Title 2-----------------------------------%
    %  \resumeSubSubheading
    %    {\textbf{Intern, Draftsman}}{\dates{May 2010}{Jun 2010}}
    %    \resumeItemListStart
    %      \resumeItem{Responsibilities}
    %        {BLAH}
    %      \resumeItem{Key contribution}
    %        {BLAH}
    %    \resumeItemListEnd

  \resumeSubHeadingListEnd

%-----------------------------------PROJECTS-----------------------------------%
\section{Key Projects}
  \resumeSubItemListStart
    \resumeSubItem{Water Coupler}
      {A \textbf{Python} HPC software for coupling multiple computational fluid
      dynamics models written in \textbf{C} (AdH), \textbf{\CC{}} (GSSHA), and
      \textbf{Fortran} (ADCIRC) for simulating compound floods due to hurricanes}
    \resumeSubItem{\href{https://github.com/gajanan-choudhary/htopy}{htopy}}
      {An \textbf{open-source Python} software for automating
      generation of Python/ctypes interfaces of C/\CC{} software for wrapping
      them in Python}
    \resumeSubItem{IR-SECT}
      {A \textbf{MATLAB} software with a graphical user interface (\textbf{GUI})
      for modeling 2D beam cross-sections that implements \textbf{graph
      algorithms} for calculating their sectorial properties to simplify stress
      analysis}
    \resumeSubItem{Machine Learning}
      {Used \textbf{image processing} and \textbf{neural networks} for detecting
      cracks in concrete surfaces}
  \resumeSubItemListEnd

%-----------------------------------EDUCATION----------------------------------%
\section{Education}
  \resumeSubHeadingListStart

    \resumeSubheading
      {The University of Texas at Austin}{Austin, TX}

      \resumeSubSubheading
        {Doctor of Philosophy (PhD) in Engineering Mechanics}{\dates{Aug 2014}{Dec 2019}}

      \resumeSubSubheading
        {Graduate Portfolio in Scientific Computing}{\dates{Aug 2014}{May 2018}}

      \resumeSubSubheading
        {Master of Science (MS) in Engineering Mechanics}{\dates{Aug 2014}{May 2017}}

    \resumeSubheading
      {Indian Institute of Technology (IIT) Kharagpur}{Kharagpur, India}

      \resumeSubSubheading
        {Bachelor of Technology (BTech) in Civil Engineering}{\dates{Jul 2009}{Jul 2013}}

  \resumeSubHeadingListEnd

%----------------------------------ACTIVITIES----------------------------------%
%\section{Certifications/Activities}
%  \resumeSubHeadingListStart
%    \resumeSubItem{Teaching}{Obtained a \textbf{basic teaching certificate} by
%                            attending teaching workshop series at UT Austin}
%    \resumeSubItem{Diversity}{Mentored a female undergraduate student in
%                             Spring '18 as part of Graduates Linked with
%                             Undergraduates in Engineering (GLUE) program at UT
%                             Austin, with the aim to inspire women to pursue
%                             graduate degrees in STEM}
%  \resumeSubHeadingListEnd

%------------------------------------THE END-----------------------------------%
\end{document}
