%--------------------------------%
% Author     : Gajanan Choudhary %
% License    : MIT               %
% About file : LaTeX Resume      %
%--------------------------------%

\documentclass[letterpaper,10pt]{article}

%--------------------------------%
% Author     : Gajanan Choudhary %
% License    : MIT               %
% About file : LaTeX Resume      %
%--------------------------------%

\RequirePackage{ifpdf}[2.3]
\ifpdf
  \pdfminorversion=7
\fi

\usepackage{cmap}  % For fixing copy pasting ligatures - ff, fi, fl, ffl, etc.
%Partially deal with overfull hbox / vbox
\usepackage[T1]{fontenc}
\usepackage[final]{microtype}
\usepackage{lmodern}
%\usepackage{ebgaramond}

% Some packages to write mathematics.
%\usepackage{amsmath,amsthm,amsfonts,amscd} 
%\usepackage{eucal} 	 	% Euler fonts
%\usepackage{epsfig}    % Allows inclusion of eps files.
%\usepackage{citesort}   % 

\usepackage{url} 		% Allows good typesetting of web URLs.

\usepackage{latexsym}
%\usepackage{graphics}
%\usepackage{epstopdf}
%\usepackage{algorithmic}
\usepackage{amssymb}
\usepackage{booktabs, makecell}
%\usepackage{bm}
%\usepackage[section]{placeins}
%\usepackage{xfp}
\usepackage{enumitem}  % For indentation of bullets

%%%%%%%%%%%%%%%%%%%%%%%%%%%%%%%%%%%%%%%%%%%%%%%%%%%%%%%%%%%%%%%%%%%%%%
\usepackage[referable]{threeparttablex}
\usepackage{footnote}
%\usepackage{cite} %% Had to comment out to prevent error
%\usepackage{siamcleveref}       %

%----------------------------------%
% Need the following for Wordcloud %
%----------------------------------%
\usepackage{graphicx}         	% Allows inclusion of eps files.
\usepackage{tikz}
\usepackage{tikzscale}
%\usepackage[percent]{overpic}
%\usetikzlibrary{calc}
%\usetikzlibrary{3d}
\usepackage{wrapfig}
\usepackage[utf8]{inputenc}
\usepackage{anyfontsize}
%\usepackage[export]{adjustbox} % Bounding box on figure
\DeclareGraphicsExtensions{.tikz,.pdf} %.eps,.png,.jpg,

\usepackage{contour}
\usepackage{ulem}

%\usepackage{csquotes}             % idk, babel/biber like it
%\usepackage[style=apa,
%backend=biber,
%sortcites=true,
%sorting=ydmdnt,
%language=american]{biblatex}      % for reference sections
%

%----------------------------------------------------------------%
% Packages borrowed from (MIT licensed) github.com/sb2nov/resume %
%----------------------------------------------------------------%
\usepackage[empty]{fullpage}
\usepackage{titlesec}
\usepackage{marvosym}
%\usepackage[usenames,dvipsnames]{color}
\usepackage{verbatim}
\usepackage{enumitem}
\usepackage[hidelinks]{hyperref}
\usepackage{fancyhdr}
\usepackage[english]{babel}
\usepackage{tabularx}
%-----------------------------------------------------------------%


%--------------------------------%
% Author     : Gajanan Choudhary %
% License    : MIT               %
% About file : LaTeX Resume      %
%--------------------------------%

%----------------------------------------------%
% Edit the following -- Put your details below %
%----------------------------------------------%

\def\Author{Gajanan Choudhary}
\def\AuthorEmail{gajanan@utexas.edu}
\def\AuthorAddress{Lake Austin Blvd, Austin, TX 78703, USA}
\def\AuthorPhoneText{\smallplus{}1 (512) 657-3030}
\def\AuthorPhoneLink{tel:15126573030}

% You need to escape special characters in website link.
% See: https://stackoverflow.com/questions/2894710/how-to-write-urls-in-latex
\def\AuthorWebsiteText{https://users.oden.utexas.edu/\textasciitilde{}gajanan/}
\def\AuthorWebsiteLink{https://users.oden.utexas.edu/~gajanan/}

%%%----------------------------------------------------------%%%
%%%----------------------------------------------------------%%%
%%% Most likely, you will not need to edit things below this %%%
%%%----------------------------------------------------------%%%
%%%----------------------------------------------------------%%%

%------------------%
% Utility commands %
%------------------%

%---------------------%
% Use sans-serif font %
%---------------------%
%\renewcommand{\familydefault}{\sfdefault}

%--------------%
% Line spacing %
%--------------%
\linespread{1.2}

%-------%
% Dates %
%-------%
\newcommand{\dates}[2]{#1 -- #2}

%---------------------------%
% For underlining hyperlink %
%---------------------------%
\renewcommand{\ULdepth}{1.8pt}
\contourlength{0.8pt}
\newcommand{\myuline}[1]{%
  \uline{\phantom{#1}}%
  \llap{\contour{white}{#1}}%
}

%-----------%
% Wordcloud %
%-----------%
\definecolor{uto}{RGB}{191, 87, 0}
\definecolor{utg}{RGB}{51, 63, 72}

% The following is needed for properly scaling text in the wordcloud. See:
% https://tex.stackexchange.com/questions/328228/appearance-of-tiny-or-scriptsize-fontsize-in-latex-horizontal-stretch/328237
%%% TEXT FONT FIX
\rmfamily
\DeclareFontShape{T1}{lmr}{bx}{sc}{<-> cmr10}{}% USE BOLD SCSHAPE NOT OTHERWISE DEFINED
%%% MATH FONT FIX
\DeclareFontFamily{OML}{zlmm}{}
\DeclareFontShape{OML}{zlmm}{m}{it}{<-> lmmi10}{}
\DeclareFontShape{OML}{zlmm}{b}{it}{<->ssub * zlmm/m/it}{}
\DeclareFontShape{OML}{zlmm}{bx}{it}{<->ssub * zlmm/m/it}{}

\DeclareMathVersion{Tinyb}
\SetSymbolFont{operators}{Tinyb}{T1}{lmr}{bx}{sc}
\SetSymbolFont{letters}{Tinyb}{OML}{zlmm}{m}{it}
%%%
\newenvironment{tinyb}{\bgroup\tiny\bfseries\scshape\mathversion{Tinyb}}{\egroup} % edit: now with \tiny included


%---------------%
% Writing "C++" %
%---------------%
% See: https://tex.stackexchange.com/questions/4302/prettiest-way-to-typeset-c-cplusplus
\def\CC{{C\nolinebreak[4]\hspace{-.05em}\raisebox{.4ex}{\tiny\bf ++}}}
% Needed for wordcloud:
\def\bigCC{{C\nolinebreak[4]\hspace{-.05em}\raisebox{.4ex}{\normalsize\bf ++}}}

%---------------------%
% Writing smaller "+" %
%---------------------%
\def\smallplus{\nolinebreak[2]\raisebox{.4ex}{\tiny\bf +}}

%---------------------------------------------------------------------------%
% Utility commands -- Borrowed from (MIT licensed) github.com/sb2nov/resume %
% Note: I have modified a few things after borrowing.                       %
%---------------------------------------------------------------------------%

\pagestyle{fancy}
\fancyhf{} % clear all header and footer fields
\fancyfoot{}
\renewcommand{\headrulewidth}{0pt}
\renewcommand{\footrulewidth}{0pt}

% Adjust margins
\addtolength{\oddsidemargin}{-0.25in}
\addtolength{\evensidemargin}{-0.25in}
\addtolength{\textwidth}{0.5in}
\addtolength{\topmargin}{-0.15in}
\addtolength{\textheight}{0.5in}

\urlstyle{same}

\raggedbottom
\raggedright
\setlength{\tabcolsep}{0in}

%---------------------------------------------------------------------------%
% Sections formatting
\titleformat{\section}{
  \vspace{-19pt}\scshape\raggedright\large
}{}{0em}{\vspace{-4pt}\color{gray}\titlerule[0.5pt]\color{black}\\}[\vspace{-6pt}]
%}{}{0em}{\color{gray}\rule{\linewidth}{0.5pt}\color{black}\\}[\vspace{-6pt}]


%---------------------------------------------------------------------------%
%---- gkc Modifying ----%
%\newcommand{\resumeSubheading}[4]{  % Original
\newcommand{\resumeSubheading}[2]{
  \vspace{-4pt}\item
    \begin{tabularx}{\linewidth}[t]{Xr}
      \textbf{#1} & #2 \\
    \end{tabularx}\vspace{-13pt}
}

\newcommand{\resumeSubSubheading}[2]{
    \vspace{-3pt}
    \begin{tabularx}{\linewidth}{Xr}
      #1 & #2 \\
    \end{tabularx}\vspace{-13pt}
}

%---------------------------------------------------------------------------%
\newcommand{\resumeItem}[2]{
  \item{
    {\textit{#1}}{: #2 \vspace{-2.5pt}}
  }
}

\newcommand{\resumeSubItem}[2]{
  \item{

    \textbf{#1}{: #2 \vspace{-7pt}} 
%    \textbf{#1}{: #2 \vspace{-9pt}}  % Use this if 11pt article
  }
}

\newcommand{\resumeUnlabeledItem}[1]{\item{{#1 \vspace{-2.5pt}}}}

%\newcommand{\resumeSubItem}[2]{\resumeItem{#1}{#2}\vspace{-4pt}}
%---------------------------------------------------------------------------%
% Bullet size and type
\renewcommand{\labelitemi}{\raisebox{1pt}{\footnotesize{$\bullet$}}}
\renewcommand{\labelitemii}{\raisebox{1pt}{\footnotesize{$\circ$}}}

%---------------------------------------------------------------------------%
\newcommand{\resumeSubHeadingListStart}{\begin{itemize}[leftmargin=0.225in]}
\newcommand{\resumeSubHeadingListEnd}{\end{itemize}\vspace{-3pt}}

%---------------------------------------------------------------------------%
\newcommand{\resumeSubItemListStart}{\vspace{-2pt}\begin{itemize}[leftmargin=0.225in]}
\newcommand{\resumeSubItemListEnd}{\end{itemize}\vspace{2pt}}

%---------------------------------------------------------------------------%
\newcommand{\resumeItemListStart}{\vspace{-7pt}\begin{itemize}[leftmargin=0.225in]}
\newcommand{\resumeItemListEnd}{\end{itemize}}

%---------------------------------------------------------------------------%
% Borrowed from:
%     https://github.com/btskinner/tex_cv
% If you want to go the partially manual route, look at another repo:
%     https://github.com/rgeirhos/academic-cv-publications

% ------------------------%
% BibLaTeX sorting scheme %
% ------------------------%

% Create a new sorting scheme that will sort in reverse chronological
% order, taking months into account correctly (i.e., in chronological
% order and not alphabetical order)

\DeclareSortingScheme{ydmdnt}{
  \sort{
    \field{presort}
  }
  \sort[final]{
    \field{sortkey}
  }
  \sort[direction=descending]{
    \field{sortyear}
    \field{year}
    \literal{9999}
  }
  \sort[direction=descending]{
    \field[padside=left,padwidth=2,padchar=0]{month}
    \literal{99}
  }
  \sort[direction=descending]{
    \field[padside=left,padwidth=2,padchar=0]{day}
    \literal{99}
  }
  \sort{
    \field{sortname}
    \field{author}
    \field{editor}
    \field{translator}
    \field{sorttitle}
    \field{title}
  }
  \sort{
    \field{sorttitle}
  }
  \sort[direction=descending]{
    \field[padside=left,padwidth=4,padchar=0]{volume}
    \literal{9999}
  }
}

%------------------%
% BibLaTeX options %
%------------------%

% language mapping
\DeclareLanguageMapping{american}{american-apa}

% ignore addendum and note fields that are sometimes useful, but not here
\DeclareSourcemap{
  \maps[datatype=bibtex]{
    \map{
      \step[fieldset=addendum, null]
      \step[fieldset=note, null]
    } 
  }
}

% macro so titles are converted to doi, url, or isbn link (if available)
% h\t https://tex.stackexchange.com/a/48409
\ExecuteBibliographyOptions{doi=false,url=false,isbn=false}
\newbibmacro{string+doiurlisbn}[1]{%
  \iffieldundef{doi}{%
    \iffieldundef{url}{%
      \iffieldundef{isbn}{%
        \iffieldundef{issn}{%
          #1%
        }{%
          \href{http://books.google.com/books?vid=ISSN\thefield{issn}}{#1}%
        }%
      }{%
        \href{http://books.google.com/books?vid=ISBN\thefield{isbn}}{#1}%
      }%
    }{%
      \href{\thefield{url}}{#1}%
    }%
  }{%
    \href{http://dx.doi.org/\thefield{doi}}{#1}%
  }%
}

\DeclareFieldFormat{title}{\usebibmacro{string+doiurlisbn}{\mkbibemph{#1}}}
\DeclareFieldFormat[article,incollection]{title}%
{\usebibmacro{string+doiurlisbn}{\mkbibquote{#1}}}

% no letter for same year entries
\defbibenvironment{mybib}
{\list
  {}
  {\setlength{\leftmargin}{\bibhang}%
    \setlength{\itemindent}{-\leftmargin}%
    \setlength{\itemsep}{\bibitemsep}%
    \setlength{\parsep}{\bibparsep}}}
{\endlist}
{\clearfield{extradate}\item}

% filter by year (only print those after year value in \recentyear
% macro above)
\defbibcheck{recent}{
  \iffieldint{year}
  {\ifnumless{\thefield{year}}{\recentyear}
    {\skipentry}
    {}}
  {\skipentry}
}



\begin{document}

%-----------------------------%
% Customize your resume below %
%-----------------------------%
{\scshape\huge\Author}\vspace{2pt}
%\href{\AuthorWebsiteLink}{\scshape\huge\Author}\vspace{2pt}

\begin{small}
  %Website: \href{\AuthorWebsiteLink}{\color[HTML]{0f4cb4}\myuline{\smash{\AuthorWebsiteText}}}
  Website: \href{\AuthorWebsiteLink}{\AuthorWebsiteText}

  \vspace{2pt}\AuthorAddress

  \vspace{1pt}\href{mailto:\AuthorEmail}{\AuthorEmail} |
  \href{\AuthorPhoneLink}{\AuthorPhoneText}
\end{small}

%------------------------------------SUMMARY-----------------------------------%
\section{Summary}
Computational scientist and software engineer with over 10 years of
inter-disciplinary research and software development experience spanning
high-performance computing (HPC), applied mathematics, computational
mechanics, performance optimization, and machine learning. Creator of 4
scientific software, contributor to 5 HPC software written in C, \CC{},
SYCL/DP\CC{}, Python, and Fortran, and author of 5 published technical
documents.
%% \\[1\baselineskip] added above since \parbox{0.546\linewidth}{\section{Summary}}
%% messes up line spacing in the new section causing overlapping lines.

%------------------------------------HEADING-----------------------------------%
\vspace{-182.8pt}

\begin{wrapfigure}[-20]{R}{0pt} %{-3cm}

  \centering

  \begin{tikzpicture}[x=1in, y=-1in, every node/.style={right}] 

    \begin{tinyb}

      \node[color=utg]at (0.20,0.13){\footnotesize Assembly\space };

      \node[color=uto]at (0.10,0.20){\tiny Kernels\space };

      \node[color=uto]at (0.57,0.25){\tiny Docker\space };

      \node[color=utg]at (0.17,0.34){\Large Python\space };

      \node[color=uto]at (0.46,0.45){\scriptsize Sparse BLAS\space };

      \node[color=utg]at (0.08,0.51){\footnotesize ANSYS\space };

      \node[color=utg]at (0.47,0.61){\normalsize MPI\space };

      \node[color=uto]at (0.09,0.92){\scriptsize Intel Corporation\space };

      \node[color=utg]at (0.12,1.03){\footnotesize SWIG\space };

      \node[color=uto]at (0.16,1.18){\small CMake\space };

      \node[color=uto]at (0.78,0.17){\large SYCL\space };

      \node[color=utg]at (0.85,0.32){\tiny GPUs\space };

      \node[color=uto]at (0.79,0.57){\footnotesize ctypes\space };

      \node[color=utg]at (0.85,0.89){\footnotesize IIT Kharagpur\space };

      \node[color=uto]at (0.50,1.03){\large CircleCI\space };

      \node[color=utg]at (0.65,1.17){\normalsize Neural Networks\space };

      \node[color=uto]at (1.28,0.14){\footnotesize Image Processing\space };

      \node[color=utg]at (1.25,0.22){\tiny Performance optimization\space };

      \node[color=uto]at (1.07,0.35){\large GitHub\space };

      \node[color=utg]at (1.10,0.50){\scriptsize HPCG\space };

      \node[color=uto]at (1.18,1.01){\footnotesize MATLAB\space };

      \node[color=uto]at (1.31,1.11){\tiny Bash\space };

      \node[color=utg]at (1.66,0.32){\footnotesize Coveralls\space };

      \node[color=uto]at (1.43,0.48){\footnotesize ADCIRC\space };

      \node[color=utg]at (1.36,0.61){\small Fortran\space };

      \node[color=utg]at (1.60,0.87){\tiny CPUs\space };

      \node[color=uto]at (1.77,0.92){\normalsize AdH\space };

      \node[color=utg]at (1.73,1.06){\small f2py\space };

      \node[color=uto]at (1.76,1.18){\scriptsize ParaView\space };

      \node[color=utg]at (2.20,0.12){\footnotesize OpenMP\space };

      \node[color=uto]at (2.08,0.23){\scriptsize Make\space };

      \node[color=uto]at (2.39,0.21){\scriptsize UT Austin\space };

      \node[color=utg]at (2.16,0.32){\small Travis CI\space };

      \node[color=utg]at (2.33,0.40){\tiny Compiler Intrinsics\space };

      \node[color=uto]at (1.93,0.50){\normalsize Applied Math\space };

      \node[color=utg]at (2.26,0.61){\scriptsize ERDC\space };

      \node[color=uto]at (2.65,0.90){\scriptsize Git\space };

      \node[color=utg]at (2.05,1.03){\huge C/\bigCC{}\space };

      \node[color=uto]at (2.40,1.19){\scriptsize SMS/WMS\space };

    \end{tinyb}

    \node[color=uto]at (0.10,0.74){\huge Computational Scientist\space };

    \draw[color=gray] (current bounding box.north east) -- (current bounding box.north west) -- (current bounding box.south west) -- (current bounding box.south east) -- cycle;

  \end{tikzpicture}

\end{wrapfigure}

\strut{} %% For ending wrapfig wrapping properly

\vspace{169.8pt}

\vspace{-2pt}
%---------------------------------WORK EXPERIENCE------------------------------%
\section{Work Experience}
  \resumeSubHeadingListStart

    %---------------------------------Title------------------------------------%
    \resumeSubheading{Software Engineer}{\dates{February 2021}{Present}}

      %-------------------------------Job 5------------------------------------%
      \resumeSubSubheading{\textbf{Intel Corporation}}{Austin, TX}
        \resumeItemListStart
          \resumeUnlabeledItem{Owned sparse linear algebra and high-performance
            conjugate gradient (HPCG) benchmark components as a contributor in
            the Intel\textsuperscript{\scriptsize{\textregistered}} oneAPI Math
            Kernel Library (oneMKL) team.}
          \resumeUnlabeledItem{Doubled the performance of sparse matrix-vector
            product (GEMV) oneMKL SYCL API on latest Intel GPUs using SIMD
            vectorization and prefetching techniques.}
          \resumeUnlabeledItem{Implemented GPU performance optimizations for
            the newly introduced sparse $\times$ sparse matrix product
            (matmat/spGEMM) SYCL API for compressed sparse row (CSR) matrices
            in oneMKL.}
          \resumeUnlabeledItem{Doubled the performance of block sparse row (BSR)
            matrix $\times$ row-major dense matrix product in oneMKL on CPUs
            by using AVX512 compiler intrinsics.}
          \resumeUnlabeledItem{Introduced optimized SYCL APIs in oneMKL
            for sorting, copying and transposing CSR matrices on GPUs.}
        \resumeItemListEnd

    %---------------------------------Title------------------------------------%
    \resumeSubheading{Research Associate}{\dates{December 2020}{February 2021}}
 
      %-------------------------------Job 4------------------------------------%
      \resumeSubSubheading{\textbf{The University of Texas at Austin}}{Austin, TX}
        \resumeItemListStart
          \resumeUnlabeledItem{Led the development of coupled PyTorch
            neural network and CFD/physics models in the Python
            software, Water Coupler, to potentially save lives and billions of
            dollars through hurricane compound flood forecasts.}
          \resumeUnlabeledItem{Led software projects that the US Army Corps
            of Engineers (USACE) sponsored to expand the capabilities of
            their CFD software, AdH, through addition of PETSc solvers and
            coupling with HEC-RAS.}
        \resumeItemListEnd

    %---------------------------------Title------------------------------------%
    \resumeSubheading{Postdoctoral Fellow}{\dates{October 2019}{December 2020}}
 
      %-------------------------------Job 3------------------------------------%
      \resumeSubSubheading{\textbf{The University of Texas at Austin}}{Austin, TX}
        \resumeItemListStart
          \resumeUnlabeledItem{Extended Water Coupler by coupling the CFD
            software, ADCIRC and GSSHA, for compound flood forecasts.}
          \resumeUnlabeledItem{Created pyADCIRC, the parallel Python interface
            of ADCIRC, to enable multi-software coupling and physics-based
            machine learning applications with it for improved hurricane
            flood forecasts.}
          \resumeUnlabeledItem{Developed, verified, and validated coupled
            diffusive wave and groundwater models in the CFD software, AdH.}
        \resumeItemListEnd

    %---------------------------------Title------------------------------------%
    \resumeSubheading{Graduate Research Assistant}{\dates{September 2014}{October 2019}}
 
    %---------------------------------Job 2------------------------------------%
      \resumeSubSubheading{\textbf{The University of Texas at Austin}}{Austin, TX}
        \resumeItemListStart
          \resumeUnlabeledItem{Led the creation of 3 new open-source software
            and contributed major modules to 2 existing HPC software through
            4 USACE-sponsored federal projects.}
          \resumeUnlabeledItem{Created the open-source parallel Python software,
            Water Coupler, by coupling 2 CFD software of USACE, AdH and GSSHA,
            to save lives through real-time hurricane compound flood forecasts.}
          \resumeUnlabeledItem{Developed the HPC Python interfaces of AdH and
            GSSHA by creating and using a new open-source software, htopy, which
            automates generation of Python wrappers of C/\CC{} software in
            hours instead of days.}
          \resumeUnlabeledItem{Developed, tested, verified, validated,
            documented, and published coupling between 2D and 3D shallow water
            models in AdH for USACE for enabling one of the first simulations of
            flooding with 3D ocean models.}
          \resumeUnlabeledItem{Created the open-source C software,
            AdH-meshbuilder, for generating arrays of coupled AdH models.}
          \resumeUnlabeledItem{Developed important physics features in AdH
            for USACE, including its wind library, 3D prism mesh adaption,
            and improved parallel load rebalancing in 3D models.}
       \resumeItemListEnd

    %---------------------------------Title------------------------------------%
    \resumeSubheading{Assistant Surveyor}{\dates{July 2013}{July 2014}}
 
      %-------------------------------Job 1------------------------------------%
      \resumeSubSubheading{\textbf{Indian Register of Shipping}}{Mumbai, India}
        \resumeItemListStart
          \resumeUnlabeledItem{Worked on 3 projects on stress response of ships
            to various loads, leading to 1 new software and 1 publication.}
          \resumeUnlabeledItem{Created and documented IR-Sect, a MATLAB software
            with a graphical user interface (GUI) for finite element (FE)
            analysis of 2D ship cross-sections, with capability to translate
            input to ANSYS.}
          \resumeUnlabeledItem{Implemented 2D modeling capabilities in IR-Sect,
            including insert, copy, move, mirror, delete, and AutoCAD-style
            selection operations to reduce modeling time to hours from days.}
          \resumeUnlabeledItem{Implemented graph algorithms in IR-Sect for
            calculating sectorial properties of ship cross-sections to enable
            shear flow, bending, torsion, and warping analysis.}
        \resumeItemListEnd
  \resumeSubHeadingListEnd
\vspace{-3pt}

%-----------------------------------PROJECTS-----------------------------------%
\section{Software Contributions}
  \resumeSubItemListStart
    \resumeSubItem{Contributor -- \href{https://https://www.intel.com/content/www/us/en/developer/tools/oneapi/onemkl.html}{Intel\textsuperscript{\scriptsize{\textregistered}} oneAPI Math Kernel Library (oneMKL)}}
      {Owned feature implementations and CPU/GPU architecture-specific
       performance optimizations for sparse linear algebra and HPCG benchmark
       components in oneMKL, written in \textbf{C}, \textbf{\CC{}},
       \textbf{SYCL/DP\CC{}}, and \textbf{Fortran}.}
    \resumeSubItem{Creator -- \href{https://github.com/gajanan-choudhary/water-coupler}{Water Coupler}}
      {Developed the \textbf{open-source parallel Python} software for coupling
      physics models, AdH (\textbf{C}), GSSHA (\textbf{\CC{}}), and ADCIRC
      (\textbf{Fortran}), as well as PyTorch machine learning/neural network
      models for real-time hurricane compound flood simulations.}
    \resumeSubItem{Contributor -- \href{https://www.erdc.usace.army.mil/Locations/CHL/AdH/}{AdH}}
      {Created the C/Python interface of AdH (adhpython), enhanced its meteorological library,
      improved mesh adaption, and coupled its shallow water, ground
      water, and diffusive wave models, all in parallel.}
    \resumeSubItem{Contributor -- \href{http://adcirc.org/}{ADCIRC}}
      {Created the Fortran/Python interface of ADCIRC (pyADCIRC), enabling physics
      based machine learning applications and multi-software coupling in Water
      Coupler.}
    \resumeSubItem{Contributor -- \href{https://www.gsshawiki.com/}{GSSHA}}
      {Created the \CC{}/Python interface of GSSHA (gsshapython) enabling
      multi-software coupling with ADCIRC and AdH in Water Coupler.}
    \resumeSubItem{Creator -- \href{https://github.com/gajanan-choudhary/htopy}{htopy}}
      {Developed an \textbf{open-source Python} software for partially
      automating the generation of Python interfaces of \textbf{C/\CC{}}
      software, and used it to generate the Python interfaces of AdH and GSSHA.}
    \resumeSubItem{Creator -- \href{https://github.com/gajanan-choudhary/AdH_meshbuilder}{AdH-meshbuilder}}
      {Developed an \textbf{open-source C} software for generating arrays of
      coupled AdH finite element models.}
    \resumeSubItem{Creator -- IR-Sect}
      {Developed a proprietary \textbf{MATLAB} software with a graphical user
      interface (\textbf{GUI}) for modeling 2D ship cross-sections and
      calculating their sectorial properties by implementing graph algorithms.}
  \resumeSubItemListEnd
\vspace{-5pt}

%------------------------------PROGRAMMING SKILLS------------------------------%
\section{Skills}
  \resumeSubHeadingListStart

    \resumeSubheading{Software development:}{}
      \resumeItemListStart
        \resumeItem{Programming}{C/\CC{}, SYCL/DP\CC{}, Python, Fortran,
            MATLAB, MPI, OpenMP, f2py, SWIG, Python/C API, and Bash.}

        \resumeItem{Tools}{GitHub, Bitbucket, Git, Mercurial, SVN,
            Travis CI, CircleCI, Docker, Coveralls, Codecov, CMake, GNU Make,
            Gcov, LCOV, GProf, GDB, Valgrind, Doxygen, \LaTeX{}, HTML,
            and CSS.}

        \resumeItem{Concepts}{Data structures, algorithms, complexity,
            object-oriented programming (OOP), standard template library (STL),
            high-performance computing, parallel programming, CPU and GPU
            performance optimization, language interoperability, debugging,
            continuous integration and continuous delivery (CI/CD), and
            test-driven development (TDD).}
      \resumeItemListEnd

    \resumeSubheading{Research:}{}
      \resumeItemListStart
        \resumeItem{Mathematics}
          {Numerical methods, partial differential equations, linear algebra,
          approximation, functional analysis, advanced theory of finite element
          methods, and optimization.}

        \resumeItem{Engineering}
          {Computational mechanics, computational fluid dynamics (CFD), solid
          mechanics, structural dynamics, fluid-structure interaction, and
          coupled numerical models.}

        \resumeItem{Applications}
          {Adaptive Hydraulics (AdH), ADvanced CIRCulation (ADCIRC), Gridded
          Surface Subsurface Hydrologic Analysis (GSSHA), ANSYS, Abaqus,
          ParaView, Aquaveo SMS/WMS, and AutoCAD.}
      \resumeItemListEnd

  \resumeSubHeadingListEnd
\vspace{-5pt}

%-----------------------------------EDUCATION----------------------------------%
\section{Education}
  \resumeSubHeadingListStart

    \resumeSubheading{Doctorate (Ph.D.) in Engineering Mechanics}{\dates{August 2014}{December 2019}}

      \resumeSubSubheading{The University of Texas at Austin}{Austin, TX}

    \resumeSubheading{Graduate Portfolio in Scientific Computing}{\dates{August 2014}{May 2018}}

      \resumeSubSubheading{The University of Texas at Austin}{Austin, TX}

    \resumeSubheading{Master of Science (M.S.) in Engineering Mechanics}{\dates{August 2014}{May 2017}}

      \resumeSubSubheading{The University of Texas at Austin}{Austin, TX}

    \resumeSubheading{Bachelor of Technology (B.Tech.) in Civil Engineering}{\dates{July 2009}{July 2013}}

      \resumeSubSubheading{Indian Institute of Technology (IIT) Kharagpur}{Kharagpur, India}

  \resumeSubHeadingListEnd
\vspace{-8pt}

%\section{Education}
%  \resumeSubHeadingListStart
%
%    \resumeSubheading{The University of Texas at Austin}{Austin, TX}
%
%      \resumeSubSubheading{Doctorate (Ph.D.) in Engineering Mechanics}{\dates{August 2014}{December 2019}}
%
%      \resumeSubSubheading{Graduate Portfolio in Scientific Computing}{\dates{August 2014}{May 2018}}
%
%      \resumeSubSubheading{Master of Science (M.S.) in Engineering Mechanics}{\dates{August 2014}{May 2017}}
%
%    \resumeSubheading{Indian Institute of Technology (IIT) Kharagpur}{Kharagpur, India}
%
%      \resumeSubSubheading{Bachelor of Technology (B.Tech.) in Civil Engineering}{\dates{July 2009}{July 2013}}
%
%  \resumeSubHeadingListEnd
%\vspace{-8pt}

%----------------------------INTERNSHIP EXPERIENCE-----------------------------%
\section{Internship Experience}
  \resumeSubHeadingListStart

    %---------------------------------Title------------------------------------%
    \resumeSubheading{Structural Design Intern}{\dates{May 2012}{July 2012}}
 
      %-------------------------------Job 2------------------------------------%
      \resumeSubSubheading{\textbf{Himanshu Tulpule and Associates}}{Pune, India}
        \resumeItemListStart
          \resumeUnlabeledItem{Designed and detailed 14 reinforced
                              concrete and steel structures spanning 7
                              different types, including trusses, frames, slabs,
                              tanks, arches, foundations, and composite beams.}
        \resumeItemListEnd

    %---------------------------------Title------------------------------------%
    \resumeSubheading{Draftsman Intern}{\dates{May 2010}{June 2010}}
 
      %-------------------------------Job 1------------------------------------%
      \resumeSubSubheading{\textbf{Himanshu Tulpule and Associates}}{Pune, India}
        \resumeItemListStart
          \resumeUnlabeledItem{Created structural drawings of reinforced
                              concrete and steel structures using AutoCAD.}
        \resumeItemListEnd
  \resumeSubHeadingListEnd
\vspace{-4pt}

%-----------------------------VOLUNTEER EXPERIENCE-----------------------------%
\section{Volunteer Experience}
  \resumeSubHeadingListStart

    %---------------------------------Title------------------------------------%
    \resumeSubheading{GLUE Mentor}{\dates{January 2018}{May 2018}}
 
      %-------------------------------Job 3------------------------------------%
      \resumeSubSubheading{\textbf{The University of Texas at Austin}}{Austin, TX}
        \resumeItemListStart
          \resumeUnlabeledItem{Mentored an undergraduate student at UT
            Austin in Spring '18 in support of the Women in Engineering
            Program's mission of inspiring women to pursue graduate degrees in
            STEM.}
          \resumeUnlabeledItem{Directed a semester-long research project for the
            mentee to train her in research through bi-weekly meetings for
            training, discussing progress, and providing guidance.}
        \resumeItemListEnd

    %---------------------------------Title------------------------------------%
    \resumeSubheading{Undergraduate Research Assistant}{\dates{March 2012}{July 2013}}
 
      %-------------------------------Job 2------------------------------------%
      \resumeSubSubheading{\textbf{Indian Institute of Technology (IIT) Kharagpur}}{Kharagpur, India}
        \resumeItemListStart
          \resumeUnlabeledItem{Published a peer-reviewed paper on detecting
            cracks in concrete using an image processing and neural network
            model that achieved 98.8\% accuracy, 99.7\% specificity,
            80\% precision, and 60\% sensitivity.}
          \resumeUnlabeledItem{Proposed improvements to genetic algorithms for
            optimization of truss weights, establishing new minima for 3
            benchmark problems in literature.}
          \resumeUnlabeledItem{Supported the research of a PhD student by
            creating finite element models and simulations of sandwich materials
            subject to under-water blasts.}
        \resumeItemListEnd

    %---------------------------------Title------------------------------------%
    \resumeSubheading{Publicity Team Member, Megalith '11}{\dates{July 2010}{December 2011}}
 
      %-------------------------------Job 1------------------------------------%
      \resumeSubSubheading{\textbf{Indian Institute of Technology (IIT) Kharagpur}}{Kharagpur, India}
        \resumeItemListStart
          \resumeUnlabeledItem{Contributed in increasing the number of
            participants in Megalith '11, the annual technical festival of the
            civil engineering department, by 25\% compared to the prior year by
            promoting the festival on social media and visiting and inviting
            students from multiple universities across India.}
          \resumeUnlabeledItem{Handled participant registration and guest
            reception during the festival, and played an advisory role in the
            team of the following year's festival, Megalith '12.}
        \resumeItemListEnd
  \resumeSubHeadingListEnd
\vspace{-4pt}

%----------------------------------ACTIVITIES----------------------------------%
\section{Activities}
  \resumeSubHeadingListStart
    \resumeSubItem{Teaching}{Obtained a \textbf{basic teaching certificate} by
                            attending a teaching workshop series at UT Austin.}
    \resumeSubItem{Dance}{Completed intermediate level training in social dance
                         forms at UT Austin and Austin Swing Syndicate.}
    \resumeSubItem{Music}{Trained in classical piano at UT Austin and performed
                         as a drummer in events at IIT Kharagpur.}
    \resumeSubItem{Theatre}{Performed in plays at IIT Kharagpur as a member of
                           the English Technology Dramatics Society.}
  \resumeSubHeadingListEnd

%------------------------------------THESES------------------------------------%
\begin{refsection}
\section{Theses}
\nocite{*}                          % cite everything
\printbibliography[heading = none,  % no heading (e.g., "References")
keyword = thesis,                   % FILTER BY < article > KEYWORD
check = recent, 
env = mybib]                        % Use mybib style
\end{refsection}
\vspace{-10pt}

%--------------------------PEER-REVIEWED PUBLICATIONS--------------------------%
\begin{refsection}
\section{Peer-reviewed Publications}
\nocite{*}                          % cite everything
\printbibliography[heading = none,  % no heading (e.g., "References")
keyword = article,                  % FILTER BY < article > KEYWORD
check = recent, 
env = mybib]                        % Use mybib style
\end{refsection}
\vspace{-10pt}

%---------------------------------OTHER REPORTS--------------------------------%
\begin{refsection}
\section{Other Reports}
\nocite{*}                          % cite everything
\printbibliography[heading = none,  % no heading (e.g., "References")
keyword = report,                   % FILTER BY < article > KEYWORD
check = recent, 
env = mybib]                        % Use mybib style
\end{refsection}
\vspace{-10pt}

%--------------------------------------TALKS-----------------------------------%
\begin{refsection}
\section{Talks}
\nocite{*}                          % cite everything
\printbibliography[heading = none,  % no heading (e.g., "References")
keyword = talk,                     % FILTER BY < article > KEYWORD
check = recent, 
env = mybib]                        % Use mybib style
\end{refsection}

%-----------------------------------SELECT WORKS-------------------------------%
%\begin{refsection}
%\section{Select Works}
%\nocite{*}                          % cite everything
%\printbibliography[heading = none,  % no heading (e.g., "References")
%keyword = select,                   % FILTER BY < article > KEYWORD
%check = recent, 
%env = mybib]                        % Use mybib style
%\end{refsection}

%------------------------------------THE END-----------------------------------%
\end{document}
