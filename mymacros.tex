%--------------------------------%
% Author     : Gajanan Choudhary %
% License    : MIT               %
% About file : LaTeX Resume      %
%--------------------------------%

%----------------------------------------------%
% Edit the following -- Put your details below %
%----------------------------------------------%

\def\Author{Gajanan Choudhary}
\def\AuthorEmail{gajanan@utexas.edu}
\def\AuthorAddress{Lake Austin Blvd, Austin, TX 78703, USA}
\def\AuthorPhoneText{\smallplus{}1 (512) 657-3030}
\def\AuthorPhoneLink{tel:15126573030}

% You need to escape special characters in website link.
% See: https://stackoverflow.com/questions/2894710/how-to-write-urls-in-latex
\def\AuthorWebsiteText{https://users.oden.utexas.edu/\textasciitilde{}gajanan/}
\def\AuthorWebsiteLink{https://users.oden.utexas.edu/~gajanan/}

%%%----------------------------------------------------------%%%
%%%----------------------------------------------------------%%%
%%% Most likely, you will not need to edit things below this %%%
%%%----------------------------------------------------------%%%
%%%----------------------------------------------------------%%%

%------------------%
% Utility commands %
%------------------%

%---------------------%
% Use sans-serif font %
%---------------------%
%\renewcommand{\familydefault}{\sfdefault}

%--------------%
% Line spacing %
%--------------%
\linespread{1.2}

%-------%
% Dates %
%-------%
\newcommand{\dates}[2]{#1 -- #2}

%---------------------------%
% For underlining hyperlink %
%---------------------------%
%\renewcommand{\ULdepth}{1.8pt}
%\contourlength{0.8pt}
%\newcommand{\myuline}[1]{%
%  \uline{\phantom{#1}}%
%  \llap{\contour{white}{#1}}%
%}

%-----------%
% Wordcloud %
%-----------%
\definecolor{uto}{RGB}{191, 87, 0}
\definecolor{utg}{RGB}{51, 63, 72}

% The following is needed for properly scaling text in the wordcloud. See:
% https://tex.stackexchange.com/questions/328228/appearance-of-tiny-or-scriptsize-fontsize-in-latex-horizontal-stretch/328237
%%% TEXT FONT FIX
\rmfamily
\DeclareFontShape{T1}{lmr}{bx}{sc}{<-> cmr10}{}% USE BOLD SCSHAPE NOT OTHERWISE DEFINED
%%% MATH FONT FIX
\DeclareFontFamily{OML}{zlmm}{}
\DeclareFontShape{OML}{zlmm}{m}{it}{<-> lmmi10}{}
\DeclareFontShape{OML}{zlmm}{b}{it}{<->ssub * zlmm/m/it}{}
\DeclareFontShape{OML}{zlmm}{bx}{it}{<->ssub * zlmm/m/it}{}

\DeclareMathVersion{Tinyb}
\SetSymbolFont{operators}{Tinyb}{T1}{lmr}{bx}{sc}
\SetSymbolFont{letters}{Tinyb}{OML}{zlmm}{m}{it}
%%%
\newenvironment{tinyb}{\bgroup\tiny\bfseries\scshape\mathversion{Tinyb}}{\egroup} % edit: now with \tiny included


%---------------%
% Writing "C++" %
%---------------%
% See: https://tex.stackexchange.com/questions/4302/prettiest-way-to-typeset-c-cplusplus
\def\CC{{C\nolinebreak[4]\hspace{-.05em}\raisebox{.4ex}{\tiny\bf ++}}}
% Needed for wordcloud:
\def\bigCC{{C\nolinebreak[4]\hspace{-.05em}\raisebox{.4ex}{\normalsize\bf ++}}}

%---------------------%
% Writing smaller "+" %
%---------------------%
\def\smallplus{\nolinebreak[2]\raisebox{.4ex}{\tiny\bf +}}

%---------------------------------------------------------------------------%
% Utility commands -- Borrowed from (MIT licensed) github.com/sb2nov/resume %
% Note: I have modified a few things after borrowing.                       %
%---------------------------------------------------------------------------%

\pagestyle{fancy}
\fancyhf{} % clear all header and footer fields
\fancyfoot{}
\renewcommand{\headrulewidth}{0pt}
\renewcommand{\footrulewidth}{0pt}

% Adjust margins
\addtolength{\oddsidemargin}{-0.25in}
\addtolength{\evensidemargin}{-0.25in}
\addtolength{\textwidth}{0.5in}
\addtolength{\topmargin}{-.05in}
\addtolength{\textheight}{0.5in}

\urlstyle{same}

\raggedbottom
\raggedright
\setlength{\tabcolsep}{0in}

%---------------------------------------------------------------------------%
% Sections formatting
\titleformat{\section}{
  \vspace{-19pt}\scshape\raggedright\large
}{}{0em}{\vspace{-4pt}\color{gray}\titlerule[0.5pt]\color{black}\\}[\vspace{-6pt}]
%}{}{0em}{\color{gray}\rule{\linewidth}{0.5pt}\color{black}\\}[\vspace{-6pt}]


%---------------------------------------------------------------------------%
%---- gkc Modifying ----%
%\newcommand{\resumeSubheading}[4]{  % Original
\newcommand{\resumeSubheading}[2]{
  \vspace{-4pt}\item
    \begin{tabularx}{\linewidth}[t]{Xr}
      \textbf{#1} & #2 \\
    \end{tabularx}\vspace{-13pt}
}

\newcommand{\resumeSubSubheading}[2]{
    \vspace{-3pt}
    \begin{tabularx}{\linewidth}{Xr}
      #1 & #2 \\
    \end{tabularx}\vspace{-13pt}
}

%---------------------------------------------------------------------------%
\newcommand{\resumeItem}[2]{
  \item{
    {\textit{#1}}{: #2 \vspace{-2.5pt}}
  }
}

\newcommand{\resumeSubItem}[2]{
  \item{

    \textbf{#1}{: #2 \vspace{-7pt}} 
%    \textbf{#1}{: #2 \vspace{-9pt}}  % Use this if 11pt article
  }
}

\newcommand{\resumeUnlabeledItem}[1]{\item{{#1 \vspace{-2.5pt}}}}

%\newcommand{\resumeSubItem}[2]{\resumeItem{#1}{#2}\vspace{-4pt}}
%---------------------------------------------------------------------------%
% Bullet size and type
\renewcommand{\labelitemi}{\raisebox{1pt}{\footnotesize{$\bullet$}}}
\renewcommand{\labelitemii}{\raisebox{1pt}{\footnotesize{$\circ$}}}

%---------------------------------------------------------------------------%
\newcommand{\resumeSubHeadingListStart}{\begin{itemize}[leftmargin=0.225in]}
\newcommand{\resumeSubHeadingListEnd}{\end{itemize}\vspace{-3pt}}

%---------------------------------------------------------------------------%
\newcommand{\resumeSubItemListStart}{\vspace{-2pt}\begin{itemize}[leftmargin=0.225in]}
\newcommand{\resumeSubItemListEnd}{\end{itemize}\vspace{2pt}}

%---------------------------------------------------------------------------%
\newcommand{\resumeItemListStart}{\vspace{-7pt}\begin{itemize}[leftmargin=0.225in]}
\newcommand{\resumeItemListEnd}{\end{itemize}}

%---------------------------------------------------------------------------%
% Borrowed from:
%     https://github.com/btskinner/tex_cv
% If you want to go the partially manual route, look at another repo:
%     https://github.com/rgeirhos/academic-cv-publications

% ------------------------%
% BibLaTeX sorting scheme %
% ------------------------%

% Create a new sorting scheme that will sort in reverse chronological
% order, taking months into account correctly (i.e., in chronological
% order and not alphabetical order)

\DeclareSortingScheme{ydmdnt}{
  \sort{
    \field{presort}
  }
  \sort[final]{
    \field{sortkey}
  }
  \sort[direction=descending]{
    \field{sortyear}
    \field{year}
    \literal{9999}
  }
  \sort[direction=descending]{
    \field[padside=left,padwidth=2,padchar=0]{month}
    \literal{99}
  }
  \sort[direction=descending]{
    \field[padside=left,padwidth=2,padchar=0]{day}
    \literal{99}
  }
  \sort{
    \field{sortname}
    \field{author}
    \field{editor}
    \field{translator}
    \field{sorttitle}
    \field{title}
  }
  \sort{
    \field{sorttitle}
  }
  \sort[direction=descending]{
    \field[padside=left,padwidth=4,padchar=0]{volume}
    \literal{9999}
  }
}

%------------------%
% BibLaTeX options %
%------------------%

% language mapping
\DeclareLanguageMapping{american}{american-apa}

% ignore addendum and note fields that are sometimes useful, but not here
\DeclareSourcemap{
  \maps[datatype=bibtex]{
    \map{
      \step[fieldset=addendum, null]
      \step[fieldset=note, null]
    } 
  }
}

% macro so titles are converted to doi, url, or isbn link (if available)
% h\t https://tex.stackexchange.com/a/48409
\ExecuteBibliographyOptions{doi=false,url=false,isbn=false}
\newbibmacro{string+doiurlisbn}[1]{%
  \iffieldundef{doi}{%
    \iffieldundef{url}{%
      \iffieldundef{isbn}{%
        \iffieldundef{issn}{%
          #1%
        }{%
          \href{http://books.google.com/books?vid=ISSN\thefield{issn}}{#1}%
        }%
      }{%
        \href{http://books.google.com/books?vid=ISBN\thefield{isbn}}{#1}%
      }%
    }{%
      \href{\thefield{url}}{#1}%
    }%
  }{%
    \href{http://dx.doi.org/\thefield{doi}}{#1}%
  }%
}

\DeclareFieldFormat{title}{\usebibmacro{string+doiurlisbn}{\mkbibemph{#1}}}
\DeclareFieldFormat[article,incollection]{title}%
{\usebibmacro{string+doiurlisbn}{\mkbibquote{#1}}}

% no letter for same year entries
\defbibenvironment{mybib}
{\list
  {}
  {\setlength{\leftmargin}{\bibhang}%
    \setlength{\itemindent}{-\leftmargin}%
    \setlength{\itemsep}{\bibitemsep}%
    \setlength{\parsep}{\bibparsep}}}
{\endlist}
{\clearfield{extradate}\item}

% filter by year (only print those after year value in \recentyear
% macro above)
\defbibcheck{recent}{
  \iffieldint{year}
  {\ifnumless{\thefield{year}}{\recentyear}
    {\skipentry}
    {}}
  {\skipentry}
}

