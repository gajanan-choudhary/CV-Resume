%--------------------------------%
% Author     : Gajanan Choudhary %
% License    : MIT               %
% About file : LaTeX Resume      %
%--------------------------------%

%----------------------------------------------%
% Edit the following -- Put your details below %
%----------------------------------------------%

\def\Author{Gajanan Choudhary}
\def\AuthorEmail{gajanan@utexas.edu}
\def\AuthorAddress{3352 Lake Austin Blvd, Austin, TX 78703}
\def\AuthorPhoneText{\smallplus{}1 (512) 657-3030}
\def\AuthorPhoneLink{tel:15126573030}

% You need to escape special characters in website link.
% See: https://stackoverflow.com/questions/2894710/how-to-write-urls-in-latex
\def\AuthorWebsiteText{https://users.oden.utexas.edu/$\smallsim$gajanan/}
\def\AuthorWebsiteLink{https://users.oden.utexas.edu/~gajanan/}

%%%----------------------------------------------------------%%%
%%%----------------------------------------------------------%%%
%%% Most likely, you will not need to edit things below this %%%
%%%----------------------------------------------------------%%%
%%%----------------------------------------------------------%%%

%------------------%
% Utility commands %
%------------------%

%---------------------%
% Use sans-serif font %
%---------------------%
%\renewcommand{\familydefault}{\sfdefault}

%--------------%
% Line spacing %
%--------------%
\linespread{1.2}

%-------%
% Dates %
%-------%
\newcommand{\dates}[2]{#1 -- #2}

%---------------------------%
% For underlining hyperlink %
%---------------------------%
\renewcommand{\ULdepth}{1.8pt}
\contourlength{0.8pt}
\newcommand{\myuline}[1]{%
  \uline{\phantom{#1}}%
  \llap{\contour{white}{#1}}%
}

%-----------%
% Wordcloud %
%-----------%
\definecolor{uto}{RGB}{191, 87, 0}
\definecolor{utg}{RGB}{51, 63, 72}

% The following is needed for properly scaling text in the wordcloud. See:
% https://tex.stackexchange.com/questions/328228/appearance-of-tiny-or-scriptsize-fontsize-in-latex-horizontal-stretch/328237
%%% TEXT FONT FIX
\rmfamily
\DeclareFontShape{T1}{lmr}{bx}{sc}{<-> cmr10}{}% USE BOLD SCSHAPE NOT OTHERWISE DEFINED
%%% MATH FONT FIX
\DeclareFontFamily{OML}{zlmm}{}
\DeclareFontShape{OML}{zlmm}{m}{it}{<-> lmmi10}{}
\DeclareFontShape{OML}{zlmm}{b}{it}{<->ssub * zlmm/m/it}{}
\DeclareFontShape{OML}{zlmm}{bx}{it}{<->ssub * zlmm/m/it}{}

\DeclareMathVersion{Tinyb}
\SetSymbolFont{operators}{Tinyb}{T1}{lmr}{bx}{sc}
\SetSymbolFont{letters}{Tinyb}{OML}{zlmm}{m}{it}
%%%
\newenvironment{tinyb}{\bgroup\tiny\bfseries\scshape\mathversion{Tinyb}}{\egroup} % edit: now with \tiny included


%---------------%
% Writing "C++" %
%---------------%
% See: https://tex.stackexchange.com/questions/4302/prettiest-way-to-typeset-c-cplusplus
\def\CC{{C\nolinebreak[4]\hspace{-.05em}\raisebox{.4ex}{\tiny\bf ++}}}
\def\bigCC{{C\nolinebreak[4]\hspace{-.05em}\raisebox{.4ex}{\normalsize\bf ++}}}

%---------------------%
% Writing smaller "~" %
%---------------------%
\def\smallplus{\nolinebreak[2]\raisebox{.4ex}{\tiny\bf +}}

%---------------------%
% Writing smaller "~" %
%---------------------%
% Following taken from:
% https://tex.stackexchange.com/questions/306079/how-to-generate-a-smaller-sim-with-respect-to-text
\makeatletter
\newcommand{\smallsym}[2]{#1{\mathpalette\make@small@sym{#2}}}
\newcommand{\make@small@sym}[2]{%
  \vcenter{\hbox{$\m@th\downgrade@style#1#2$}}%
}
\newcommand{\downgrade@style}[1]{%
  \ifx#1\displaystyle\scriptstyle\else
    \ifx#1\textstyle\scriptstyle\else
      \scriptscriptstyle
  \fi\fi
}
\makeatother

% This allows the \smallsim command for \sim.
\newcommand{\smallsim}{\smallsym{\mathrel}{\sim}}
%--------------------------------------------------------------%


%---------------------------------------------------------------------------%
% Utility commands -- Borrowed from (MIT licensed) github.com/sb2nov/resume %
% Note: I have modified a few things after borrowing.                       %
%---------------------------------------------------------------------------%

\pagestyle{fancy}
\fancyhf{} % clear all header and footer fields
\fancyfoot{}
\renewcommand{\headrulewidth}{0pt}
\renewcommand{\footrulewidth}{0pt}

% Adjust margins
\addtolength{\oddsidemargin}{-0.25in}
\addtolength{\evensidemargin}{-0.25in}
\addtolength{\textwidth}{0.5in}
\addtolength{\topmargin}{-.25in}
\addtolength{\textheight}{0.5in}

\urlstyle{same}

\raggedbottom
\raggedright
\setlength{\tabcolsep}{0in}

%---------------------------------------------------------------------------%
% Sections formatting
\titleformat{\section}{
  \vspace{-19pt}\scshape\raggedright\large
}{}{0em}{\vspace{-4pt}\color{gray}\titlerule[0.5pt]\color{black}\\}[\vspace{-6pt}]
%}{}{0em}{\color{gray}\rule{\linewidth}{0.5pt}\color{black}\\}[\vspace{-6pt}]


%---------------------------------------------------------------------------%
%---- gkc Modifying ----%
%\newcommand{\resumeSubheading}[4]{  % Original
\newcommand{\resumeSubheading}[2]{
  \vspace{-4pt}\item
    \begin{tabularx}{\linewidth}[t]{Xr}
      \textbf{#1} & #2 \\
    \end{tabularx}\vspace{-13pt}
}

\newcommand{\resumeSubSubheading}[2]{
    \vspace{-3pt}
    \begin{tabularx}{\linewidth}{Xr}
      #1 & #2 \\
    \end{tabularx}\vspace{-13pt}
}

%---------------------------------------------------------------------------%
\newcommand{\resumeItem}[2]{
  \item{
    {#1}{: #2 \vspace{-2.5pt}}
  }
}
\newcommand{\resumeSubItem}[2]{
  \item{

    \textbf{#1}{: #2 \vspace{-7pt}} 
%    \textbf{#1}{: #2 \vspace{-9pt}}  % Use this if 11pt article
  }
}

%\newcommand{\resumeSubItem}[2]{\resumeItem{#1}{#2}\vspace{-4pt}}
%---------------------------------------------------------------------------%
% Bullet size and type
\renewcommand{\labelitemi}{\raisebox{1pt}{\footnotesize{$\bullet$}}}
\renewcommand{\labelitemii}{\raisebox{1pt}{\footnotesize{$\circ$}}}

%---------------------------------------------------------------------------%
\newcommand{\resumeSubHeadingListStart}{\begin{itemize}[leftmargin=0.225in]}
\newcommand{\resumeSubHeadingListEnd}{\end{itemize}}

%---------------------------------------------------------------------------%
\newcommand{\resumeSubItemListStart}{\vspace{-2pt}\begin{itemize}[leftmargin=0.225in]}
\newcommand{\resumeSubItemListEnd}{\end{itemize}\vspace{-1pt}}

%---------------------------------------------------------------------------%
\newcommand{\resumeItemListStart}{\vspace{-7pt}\begin{itemize}[leftmargin=0.225in]}
\newcommand{\resumeItemListEnd}{\end{itemize}}

